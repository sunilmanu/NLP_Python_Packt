
% Default to the notebook output style

    


% Inherit from the specified cell style.




    
\documentclass[11pt]{article}

    
    
    \usepackage[T1]{fontenc}
    % Nicer default font (+ math font) than Computer Modern for most use cases
    \usepackage{mathpazo}

    % Basic figure setup, for now with no caption control since it's done
    % automatically by Pandoc (which extracts ![](path) syntax from Markdown).
    \usepackage{graphicx}
    % We will generate all images so they have a width \maxwidth. This means
    % that they will get their normal width if they fit onto the page, but
    % are scaled down if they would overflow the margins.
    \makeatletter
    \def\maxwidth{\ifdim\Gin@nat@width>\linewidth\linewidth
    \else\Gin@nat@width\fi}
    \makeatother
    \let\Oldincludegraphics\includegraphics
    % Set max figure width to be 80% of text width, for now hardcoded.
    \renewcommand{\includegraphics}[1]{\Oldincludegraphics[width=.8\maxwidth]{#1}}
    % Ensure that by default, figures have no caption (until we provide a
    % proper Figure object with a Caption API and a way to capture that
    % in the conversion process - todo).
    \usepackage{caption}
    \DeclareCaptionLabelFormat{nolabel}{}
    \captionsetup{labelformat=nolabel}

    \usepackage{adjustbox} % Used to constrain images to a maximum size 
    \usepackage{xcolor} % Allow colors to be defined
    \usepackage{enumerate} % Needed for markdown enumerations to work
    \usepackage{geometry} % Used to adjust the document margins
    \usepackage{amsmath} % Equations
    \usepackage{amssymb} % Equations
    \usepackage{textcomp} % defines textquotesingle
    % Hack from http://tex.stackexchange.com/a/47451/13684:
    \AtBeginDocument{%
        \def\PYZsq{\textquotesingle}% Upright quotes in Pygmentized code
    }
    \usepackage{upquote} % Upright quotes for verbatim code
    \usepackage{eurosym} % defines \euro
    \usepackage[mathletters]{ucs} % Extended unicode (utf-8) support
    \usepackage[utf8x]{inputenc} % Allow utf-8 characters in the tex document
    \usepackage{fancyvrb} % verbatim replacement that allows latex
    \usepackage{grffile} % extends the file name processing of package graphics 
                         % to support a larger range 
    % The hyperref package gives us a pdf with properly built
    % internal navigation ('pdf bookmarks' for the table of contents,
    % internal cross-reference links, web links for URLs, etc.)
    \usepackage{hyperref}
    \usepackage{longtable} % longtable support required by pandoc >1.10
    \usepackage{booktabs}  % table support for pandoc > 1.12.2
    \usepackage[inline]{enumitem} % IRkernel/repr support (it uses the enumerate* environment)
    \usepackage[normalem]{ulem} % ulem is needed to support strikethroughs (\sout)
                                % normalem makes italics be italics, not underlines
    

    
    
    % Colors for the hyperref package
    \definecolor{urlcolor}{rgb}{0,.145,.698}
    \definecolor{linkcolor}{rgb}{.71,0.21,0.01}
    \definecolor{citecolor}{rgb}{.12,.54,.11}

    % ANSI colors
    \definecolor{ansi-black}{HTML}{3E424D}
    \definecolor{ansi-black-intense}{HTML}{282C36}
    \definecolor{ansi-red}{HTML}{E75C58}
    \definecolor{ansi-red-intense}{HTML}{B22B31}
    \definecolor{ansi-green}{HTML}{00A250}
    \definecolor{ansi-green-intense}{HTML}{007427}
    \definecolor{ansi-yellow}{HTML}{DDB62B}
    \definecolor{ansi-yellow-intense}{HTML}{B27D12}
    \definecolor{ansi-blue}{HTML}{208FFB}
    \definecolor{ansi-blue-intense}{HTML}{0065CA}
    \definecolor{ansi-magenta}{HTML}{D160C4}
    \definecolor{ansi-magenta-intense}{HTML}{A03196}
    \definecolor{ansi-cyan}{HTML}{60C6C8}
    \definecolor{ansi-cyan-intense}{HTML}{258F8F}
    \definecolor{ansi-white}{HTML}{C5C1B4}
    \definecolor{ansi-white-intense}{HTML}{A1A6B2}

    % commands and environments needed by pandoc snippets
    % extracted from the output of `pandoc -s`
    \providecommand{\tightlist}{%
      \setlength{\itemsep}{0pt}\setlength{\parskip}{0pt}}
    \DefineVerbatimEnvironment{Highlighting}{Verbatim}{commandchars=\\\{\}}
    % Add ',fontsize=\small' for more characters per line
    \newenvironment{Shaded}{}{}
    \newcommand{\KeywordTok}[1]{\textcolor[rgb]{0.00,0.44,0.13}{\textbf{{#1}}}}
    \newcommand{\DataTypeTok}[1]{\textcolor[rgb]{0.56,0.13,0.00}{{#1}}}
    \newcommand{\DecValTok}[1]{\textcolor[rgb]{0.25,0.63,0.44}{{#1}}}
    \newcommand{\BaseNTok}[1]{\textcolor[rgb]{0.25,0.63,0.44}{{#1}}}
    \newcommand{\FloatTok}[1]{\textcolor[rgb]{0.25,0.63,0.44}{{#1}}}
    \newcommand{\CharTok}[1]{\textcolor[rgb]{0.25,0.44,0.63}{{#1}}}
    \newcommand{\StringTok}[1]{\textcolor[rgb]{0.25,0.44,0.63}{{#1}}}
    \newcommand{\CommentTok}[1]{\textcolor[rgb]{0.38,0.63,0.69}{\textit{{#1}}}}
    \newcommand{\OtherTok}[1]{\textcolor[rgb]{0.00,0.44,0.13}{{#1}}}
    \newcommand{\AlertTok}[1]{\textcolor[rgb]{1.00,0.00,0.00}{\textbf{{#1}}}}
    \newcommand{\FunctionTok}[1]{\textcolor[rgb]{0.02,0.16,0.49}{{#1}}}
    \newcommand{\RegionMarkerTok}[1]{{#1}}
    \newcommand{\ErrorTok}[1]{\textcolor[rgb]{1.00,0.00,0.00}{\textbf{{#1}}}}
    \newcommand{\NormalTok}[1]{{#1}}
    
    % Additional commands for more recent versions of Pandoc
    \newcommand{\ConstantTok}[1]{\textcolor[rgb]{0.53,0.00,0.00}{{#1}}}
    \newcommand{\SpecialCharTok}[1]{\textcolor[rgb]{0.25,0.44,0.63}{{#1}}}
    \newcommand{\VerbatimStringTok}[1]{\textcolor[rgb]{0.25,0.44,0.63}{{#1}}}
    \newcommand{\SpecialStringTok}[1]{\textcolor[rgb]{0.73,0.40,0.53}{{#1}}}
    \newcommand{\ImportTok}[1]{{#1}}
    \newcommand{\DocumentationTok}[1]{\textcolor[rgb]{0.73,0.13,0.13}{\textit{{#1}}}}
    \newcommand{\AnnotationTok}[1]{\textcolor[rgb]{0.38,0.63,0.69}{\textbf{\textit{{#1}}}}}
    \newcommand{\CommentVarTok}[1]{\textcolor[rgb]{0.38,0.63,0.69}{\textbf{\textit{{#1}}}}}
    \newcommand{\VariableTok}[1]{\textcolor[rgb]{0.10,0.09,0.49}{{#1}}}
    \newcommand{\ControlFlowTok}[1]{\textcolor[rgb]{0.00,0.44,0.13}{\textbf{{#1}}}}
    \newcommand{\OperatorTok}[1]{\textcolor[rgb]{0.40,0.40,0.40}{{#1}}}
    \newcommand{\BuiltInTok}[1]{{#1}}
    \newcommand{\ExtensionTok}[1]{{#1}}
    \newcommand{\PreprocessorTok}[1]{\textcolor[rgb]{0.74,0.48,0.00}{{#1}}}
    \newcommand{\AttributeTok}[1]{\textcolor[rgb]{0.49,0.56,0.16}{{#1}}}
    \newcommand{\InformationTok}[1]{\textcolor[rgb]{0.38,0.63,0.69}{\textbf{\textit{{#1}}}}}
    \newcommand{\WarningTok}[1]{\textcolor[rgb]{0.38,0.63,0.69}{\textbf{\textit{{#1}}}}}
    
    
    % Define a nice break command that doesn't care if a line doesn't already
    % exist.
    \def\br{\hspace*{\fill} \\* }
    % Math Jax compatability definitions
    \def\gt{>}
    \def\lt{<}
    % Document parameters
    \title{Chapter 3 - Morphology ? Getting Our Feet Wet}
    
    
    

    % Pygments definitions
    
\makeatletter
\def\PY@reset{\let\PY@it=\relax \let\PY@bf=\relax%
    \let\PY@ul=\relax \let\PY@tc=\relax%
    \let\PY@bc=\relax \let\PY@ff=\relax}
\def\PY@tok#1{\csname PY@tok@#1\endcsname}
\def\PY@toks#1+{\ifx\relax#1\empty\else%
    \PY@tok{#1}\expandafter\PY@toks\fi}
\def\PY@do#1{\PY@bc{\PY@tc{\PY@ul{%
    \PY@it{\PY@bf{\PY@ff{#1}}}}}}}
\def\PY#1#2{\PY@reset\PY@toks#1+\relax+\PY@do{#2}}

\expandafter\def\csname PY@tok@w\endcsname{\def\PY@tc##1{\textcolor[rgb]{0.73,0.73,0.73}{##1}}}
\expandafter\def\csname PY@tok@c\endcsname{\let\PY@it=\textit\def\PY@tc##1{\textcolor[rgb]{0.25,0.50,0.50}{##1}}}
\expandafter\def\csname PY@tok@cp\endcsname{\def\PY@tc##1{\textcolor[rgb]{0.74,0.48,0.00}{##1}}}
\expandafter\def\csname PY@tok@k\endcsname{\let\PY@bf=\textbf\def\PY@tc##1{\textcolor[rgb]{0.00,0.50,0.00}{##1}}}
\expandafter\def\csname PY@tok@kp\endcsname{\def\PY@tc##1{\textcolor[rgb]{0.00,0.50,0.00}{##1}}}
\expandafter\def\csname PY@tok@kt\endcsname{\def\PY@tc##1{\textcolor[rgb]{0.69,0.00,0.25}{##1}}}
\expandafter\def\csname PY@tok@o\endcsname{\def\PY@tc##1{\textcolor[rgb]{0.40,0.40,0.40}{##1}}}
\expandafter\def\csname PY@tok@ow\endcsname{\let\PY@bf=\textbf\def\PY@tc##1{\textcolor[rgb]{0.67,0.13,1.00}{##1}}}
\expandafter\def\csname PY@tok@nb\endcsname{\def\PY@tc##1{\textcolor[rgb]{0.00,0.50,0.00}{##1}}}
\expandafter\def\csname PY@tok@nf\endcsname{\def\PY@tc##1{\textcolor[rgb]{0.00,0.00,1.00}{##1}}}
\expandafter\def\csname PY@tok@nc\endcsname{\let\PY@bf=\textbf\def\PY@tc##1{\textcolor[rgb]{0.00,0.00,1.00}{##1}}}
\expandafter\def\csname PY@tok@nn\endcsname{\let\PY@bf=\textbf\def\PY@tc##1{\textcolor[rgb]{0.00,0.00,1.00}{##1}}}
\expandafter\def\csname PY@tok@ne\endcsname{\let\PY@bf=\textbf\def\PY@tc##1{\textcolor[rgb]{0.82,0.25,0.23}{##1}}}
\expandafter\def\csname PY@tok@nv\endcsname{\def\PY@tc##1{\textcolor[rgb]{0.10,0.09,0.49}{##1}}}
\expandafter\def\csname PY@tok@no\endcsname{\def\PY@tc##1{\textcolor[rgb]{0.53,0.00,0.00}{##1}}}
\expandafter\def\csname PY@tok@nl\endcsname{\def\PY@tc##1{\textcolor[rgb]{0.63,0.63,0.00}{##1}}}
\expandafter\def\csname PY@tok@ni\endcsname{\let\PY@bf=\textbf\def\PY@tc##1{\textcolor[rgb]{0.60,0.60,0.60}{##1}}}
\expandafter\def\csname PY@tok@na\endcsname{\def\PY@tc##1{\textcolor[rgb]{0.49,0.56,0.16}{##1}}}
\expandafter\def\csname PY@tok@nt\endcsname{\let\PY@bf=\textbf\def\PY@tc##1{\textcolor[rgb]{0.00,0.50,0.00}{##1}}}
\expandafter\def\csname PY@tok@nd\endcsname{\def\PY@tc##1{\textcolor[rgb]{0.67,0.13,1.00}{##1}}}
\expandafter\def\csname PY@tok@s\endcsname{\def\PY@tc##1{\textcolor[rgb]{0.73,0.13,0.13}{##1}}}
\expandafter\def\csname PY@tok@sd\endcsname{\let\PY@it=\textit\def\PY@tc##1{\textcolor[rgb]{0.73,0.13,0.13}{##1}}}
\expandafter\def\csname PY@tok@si\endcsname{\let\PY@bf=\textbf\def\PY@tc##1{\textcolor[rgb]{0.73,0.40,0.53}{##1}}}
\expandafter\def\csname PY@tok@se\endcsname{\let\PY@bf=\textbf\def\PY@tc##1{\textcolor[rgb]{0.73,0.40,0.13}{##1}}}
\expandafter\def\csname PY@tok@sr\endcsname{\def\PY@tc##1{\textcolor[rgb]{0.73,0.40,0.53}{##1}}}
\expandafter\def\csname PY@tok@ss\endcsname{\def\PY@tc##1{\textcolor[rgb]{0.10,0.09,0.49}{##1}}}
\expandafter\def\csname PY@tok@sx\endcsname{\def\PY@tc##1{\textcolor[rgb]{0.00,0.50,0.00}{##1}}}
\expandafter\def\csname PY@tok@m\endcsname{\def\PY@tc##1{\textcolor[rgb]{0.40,0.40,0.40}{##1}}}
\expandafter\def\csname PY@tok@gh\endcsname{\let\PY@bf=\textbf\def\PY@tc##1{\textcolor[rgb]{0.00,0.00,0.50}{##1}}}
\expandafter\def\csname PY@tok@gu\endcsname{\let\PY@bf=\textbf\def\PY@tc##1{\textcolor[rgb]{0.50,0.00,0.50}{##1}}}
\expandafter\def\csname PY@tok@gd\endcsname{\def\PY@tc##1{\textcolor[rgb]{0.63,0.00,0.00}{##1}}}
\expandafter\def\csname PY@tok@gi\endcsname{\def\PY@tc##1{\textcolor[rgb]{0.00,0.63,0.00}{##1}}}
\expandafter\def\csname PY@tok@gr\endcsname{\def\PY@tc##1{\textcolor[rgb]{1.00,0.00,0.00}{##1}}}
\expandafter\def\csname PY@tok@ge\endcsname{\let\PY@it=\textit}
\expandafter\def\csname PY@tok@gs\endcsname{\let\PY@bf=\textbf}
\expandafter\def\csname PY@tok@gp\endcsname{\let\PY@bf=\textbf\def\PY@tc##1{\textcolor[rgb]{0.00,0.00,0.50}{##1}}}
\expandafter\def\csname PY@tok@go\endcsname{\def\PY@tc##1{\textcolor[rgb]{0.53,0.53,0.53}{##1}}}
\expandafter\def\csname PY@tok@gt\endcsname{\def\PY@tc##1{\textcolor[rgb]{0.00,0.27,0.87}{##1}}}
\expandafter\def\csname PY@tok@err\endcsname{\def\PY@bc##1{\setlength{\fboxsep}{0pt}\fcolorbox[rgb]{1.00,0.00,0.00}{1,1,1}{\strut ##1}}}
\expandafter\def\csname PY@tok@kc\endcsname{\let\PY@bf=\textbf\def\PY@tc##1{\textcolor[rgb]{0.00,0.50,0.00}{##1}}}
\expandafter\def\csname PY@tok@kd\endcsname{\let\PY@bf=\textbf\def\PY@tc##1{\textcolor[rgb]{0.00,0.50,0.00}{##1}}}
\expandafter\def\csname PY@tok@kn\endcsname{\let\PY@bf=\textbf\def\PY@tc##1{\textcolor[rgb]{0.00,0.50,0.00}{##1}}}
\expandafter\def\csname PY@tok@kr\endcsname{\let\PY@bf=\textbf\def\PY@tc##1{\textcolor[rgb]{0.00,0.50,0.00}{##1}}}
\expandafter\def\csname PY@tok@bp\endcsname{\def\PY@tc##1{\textcolor[rgb]{0.00,0.50,0.00}{##1}}}
\expandafter\def\csname PY@tok@fm\endcsname{\def\PY@tc##1{\textcolor[rgb]{0.00,0.00,1.00}{##1}}}
\expandafter\def\csname PY@tok@vc\endcsname{\def\PY@tc##1{\textcolor[rgb]{0.10,0.09,0.49}{##1}}}
\expandafter\def\csname PY@tok@vg\endcsname{\def\PY@tc##1{\textcolor[rgb]{0.10,0.09,0.49}{##1}}}
\expandafter\def\csname PY@tok@vi\endcsname{\def\PY@tc##1{\textcolor[rgb]{0.10,0.09,0.49}{##1}}}
\expandafter\def\csname PY@tok@vm\endcsname{\def\PY@tc##1{\textcolor[rgb]{0.10,0.09,0.49}{##1}}}
\expandafter\def\csname PY@tok@sa\endcsname{\def\PY@tc##1{\textcolor[rgb]{0.73,0.13,0.13}{##1}}}
\expandafter\def\csname PY@tok@sb\endcsname{\def\PY@tc##1{\textcolor[rgb]{0.73,0.13,0.13}{##1}}}
\expandafter\def\csname PY@tok@sc\endcsname{\def\PY@tc##1{\textcolor[rgb]{0.73,0.13,0.13}{##1}}}
\expandafter\def\csname PY@tok@dl\endcsname{\def\PY@tc##1{\textcolor[rgb]{0.73,0.13,0.13}{##1}}}
\expandafter\def\csname PY@tok@s2\endcsname{\def\PY@tc##1{\textcolor[rgb]{0.73,0.13,0.13}{##1}}}
\expandafter\def\csname PY@tok@sh\endcsname{\def\PY@tc##1{\textcolor[rgb]{0.73,0.13,0.13}{##1}}}
\expandafter\def\csname PY@tok@s1\endcsname{\def\PY@tc##1{\textcolor[rgb]{0.73,0.13,0.13}{##1}}}
\expandafter\def\csname PY@tok@mb\endcsname{\def\PY@tc##1{\textcolor[rgb]{0.40,0.40,0.40}{##1}}}
\expandafter\def\csname PY@tok@mf\endcsname{\def\PY@tc##1{\textcolor[rgb]{0.40,0.40,0.40}{##1}}}
\expandafter\def\csname PY@tok@mh\endcsname{\def\PY@tc##1{\textcolor[rgb]{0.40,0.40,0.40}{##1}}}
\expandafter\def\csname PY@tok@mi\endcsname{\def\PY@tc##1{\textcolor[rgb]{0.40,0.40,0.40}{##1}}}
\expandafter\def\csname PY@tok@il\endcsname{\def\PY@tc##1{\textcolor[rgb]{0.40,0.40,0.40}{##1}}}
\expandafter\def\csname PY@tok@mo\endcsname{\def\PY@tc##1{\textcolor[rgb]{0.40,0.40,0.40}{##1}}}
\expandafter\def\csname PY@tok@ch\endcsname{\let\PY@it=\textit\def\PY@tc##1{\textcolor[rgb]{0.25,0.50,0.50}{##1}}}
\expandafter\def\csname PY@tok@cm\endcsname{\let\PY@it=\textit\def\PY@tc##1{\textcolor[rgb]{0.25,0.50,0.50}{##1}}}
\expandafter\def\csname PY@tok@cpf\endcsname{\let\PY@it=\textit\def\PY@tc##1{\textcolor[rgb]{0.25,0.50,0.50}{##1}}}
\expandafter\def\csname PY@tok@c1\endcsname{\let\PY@it=\textit\def\PY@tc##1{\textcolor[rgb]{0.25,0.50,0.50}{##1}}}
\expandafter\def\csname PY@tok@cs\endcsname{\let\PY@it=\textit\def\PY@tc##1{\textcolor[rgb]{0.25,0.50,0.50}{##1}}}

\def\PYZbs{\char`\\}
\def\PYZus{\char`\_}
\def\PYZob{\char`\{}
\def\PYZcb{\char`\}}
\def\PYZca{\char`\^}
\def\PYZam{\char`\&}
\def\PYZlt{\char`\<}
\def\PYZgt{\char`\>}
\def\PYZsh{\char`\#}
\def\PYZpc{\char`\%}
\def\PYZdl{\char`\$}
\def\PYZhy{\char`\-}
\def\PYZsq{\char`\'}
\def\PYZdq{\char`\"}
\def\PYZti{\char`\~}
% for compatibility with earlier versions
\def\PYZat{@}
\def\PYZlb{[}
\def\PYZrb{]}
\makeatother


    % Exact colors from NB
    \definecolor{incolor}{rgb}{0.0, 0.0, 0.5}
    \definecolor{outcolor}{rgb}{0.545, 0.0, 0.0}



    
    % Prevent overflowing lines due to hard-to-break entities
    \sloppy 
    % Setup hyperref package
    \hypersetup{
      breaklinks=true,  % so long urls are correctly broken across lines
      colorlinks=true,
      urlcolor=urlcolor,
      linkcolor=linkcolor,
      citecolor=citecolor,
      }
    % Slightly bigger margins than the latex defaults
    
    \geometry{verbose,tmargin=1in,bmargin=1in,lmargin=1in,rmargin=1in}
    
    

    \begin{document}
    
    
    \maketitle
    
    

    
    \hypertarget{morphology-getting-our-feet-wet}{%
\section{Morphology -- Getting Our Feet
Wet}\label{morphology-getting-our-feet-wet}}

    Morphology may be defined as the study of the composition of words using
morphemes. A morpheme is the smallest unit of language that has meaning.
In this chapter, we will discuss stemming and lemmatizing, stemmer and
lemmatizer for non-English languages, developing a morphological
analyzer and morphological generator using machine learning tools,
search engines, and many such concepts.
In brief, this chapter will include the following topics:
•Introducing morphology
•Understanding stemmer
•Understanding lemmatization
•Developing a stemmer for non-English languages
•Morphological analyzer
•Morphological generator
•Search engine
    \hypertarget{introducing-morphology}{%
\section{Introducing morphology}\label{introducing-morphology}}

    Morphology may be defined as the study of the production of tokens with
the help of morphemes. A morpheme is the basic unit of language carrying
meaning. There are two types of morpheme: stems and affixes (suffixes,
prefixes, infixes, and circumfixes).

    Stems are also referred to as free morphemes, since they can even exist
without adding affixes. Affixes are referred to as bound morphemes,
since they cannot exist in a free form and they always exist along with
free morphemes. Consider the word unbelievable. Here, believe is a stem
or a free morpheme. It can exist on its own. The morphemes un and able
are affixes or bound morphemes. They cannot exist in a free form, but
they exist together with stem.

    There are three kinds of language, namely isolating languages,
agglutinative languages, and inflecting languages. Morphology has a
different meaning in all these languages. Isolating languages are those
languages in which words are merely free morphemes and they do not carry
any tense (past, present, and future) and number (singular or plural)
information. Mandarin Chinese is an example of an isolating language.

    Agglutinative languages are those in which small words combine together
to convey compound information. Turkish is an example of an
agglutinative language.

    Inflecting languages are those in which words are broken down into
simpler units, but all these simpler units exhibit different meanings.
Latin is an example of an inflecting language

    Morphological processes are of the following types: inflection,
derivation, semiaffixes and combining forms, and cliticization.
Inflection means transforming the word into a form so that it represents
person, number, tense, gender, case, aspect, and mood. Here, the
syntactic category of a token remains the same. In derivation, the
syntactic category of a word is also changed. Semiaffixes are bound
morphemes that exhibit words, such as quality, for example, noteworthy,
antisocial, anticlockwise, and so on.

    \hypertarget{understanding-stemmer}{%
\section{Understanding stemmer}\label{understanding-stemmer}}

    Stemming may be defined as the process of obtaining a stem from a word
by eliminating the affixes from a word. For example, in the case of the
word raining, stemmer would return the root word or stem word rain by
removing the affix from raining. In order to increase the accuracy of
information retrieval, search engines mostly use stemming to get the
stems and store them as indexed words. Search engines call words with
the same meaning synonyms, which may be a kind of query expansion known
as conflation. Martin Porter has designed a well-known stemming
algorithm known as the Porter stemming algorithm. This algorithm is
basically designed to replace and eliminate some well-known suffices
present in English words. To perform stemming in NLTK, we can simply do
an instantiation of the PorterStemmer class and then perform stemming by
calling the stem method.

    \begin{Verbatim}[commandchars=\\\{\}]
{\color{incolor}In [{\color{incolor}2}]:} \PY{k+kn}{import} \PY{n+nn}{nltk}
        \PY{k+kn}{from} \PY{n+nn}{nltk}\PY{n+nn}{.}\PY{n+nn}{stem} \PY{k}{import} \PY{n}{PorterStemmer}
        \PY{n}{stemmerporter} \PY{o}{=} \PY{n}{PorterStemmer}\PY{p}{(}\PY{p}{)}
        \PY{n}{stemmerporter}\PY{o}{.}\PY{n}{stem}\PY{p}{(}\PY{l+s+s1}{\PYZsq{}}\PY{l+s+s1}{working}\PY{l+s+s1}{\PYZsq{}}\PY{p}{)}
\end{Verbatim}


\begin{Verbatim}[commandchars=\\\{\}]
{\color{outcolor}Out[{\color{outcolor}2}]:} 'work'
\end{Verbatim}
            
    \begin{Verbatim}[commandchars=\\\{\}]
{\color{incolor}In [{\color{incolor}3}]:} \PY{n}{stemmerporter}\PY{o}{.}\PY{n}{stem}\PY{p}{(}\PY{l+s+s1}{\PYZsq{}}\PY{l+s+s1}{happiness}\PY{l+s+s1}{\PYZsq{}}\PY{p}{)}
\end{Verbatim}


\begin{Verbatim}[commandchars=\\\{\}]
{\color{outcolor}Out[{\color{outcolor}3}]:} 'happi'
\end{Verbatim}
            
    Types of Stemmers: i)PorterStemmer ii)LancasterStemmer iii)RegExp
Stemmer iv)SnowballStemmer

    \hypertarget{lancasterstemmer}{%
\section{LancasterStemmer}\label{lancasterstemmer}}

Lancaster stemming algorithm was introduced at Lancaster University.
Similar to the PorterStemmer class, the LancasterStemmer class is used
in NLTK to implement Lancaster stemming. However, one of the major
differences between the two algorithms is that Lancaster stemming
involves the use of more words of different sentiments as compared to
Porter Stemming.

    \begin{Verbatim}[commandchars=\\\{\}]
{\color{incolor}In [{\color{incolor}5}]:} \PY{k+kn}{from} \PY{n+nn}{nltk}\PY{n+nn}{.}\PY{n+nn}{stem} \PY{k}{import} \PY{n}{LancasterStemmer}
        \PY{n}{stemmer\PYZus{}lan}\PY{o}{=}\PY{n}{LancasterStemmer}\PY{p}{(}\PY{p}{)}
        \PY{n}{stemmer\PYZus{}lan}\PY{o}{.}\PY{n}{stem}\PY{p}{(}\PY{l+s+s1}{\PYZsq{}}\PY{l+s+s1}{working}\PY{l+s+s1}{\PYZsq{}}\PY{p}{)}
\end{Verbatim}


\begin{Verbatim}[commandchars=\\\{\}]
{\color{outcolor}Out[{\color{outcolor}5}]:} 'work'
\end{Verbatim}
            
    \begin{Verbatim}[commandchars=\\\{\}]
{\color{incolor}In [{\color{incolor}8}]:} \PY{n}{stemmer\PYZus{}lan}\PY{o}{.}\PY{n}{stem}\PY{p}{(}\PY{l+s+s1}{\PYZsq{}}\PY{l+s+s1}{happiness}\PY{l+s+s1}{\PYZsq{}}\PY{p}{)}
\end{Verbatim}


\begin{Verbatim}[commandchars=\\\{\}]
{\color{outcolor}Out[{\color{outcolor}8}]:} 'happy'
\end{Verbatim}
            
    \hypertarget{regexstemmer}{%
\section{RegexStemmer}\label{regexstemmer}}

We can also build our own stemmer in NLTK using RegexpStemmer. It works
by accepting a string and eliminating the string from the prefix or
suffix of a word when a match is found.

    \begin{Verbatim}[commandchars=\\\{\}]
{\color{incolor}In [{\color{incolor}9}]:} \PY{k+kn}{from} \PY{n+nn}{nltk}\PY{n+nn}{.}\PY{n+nn}{stem} \PY{k}{import} \PY{n}{RegexpStemmer}
        \PY{n}{stemmer\PYZus{}regexp}\PY{o}{=}\PY{n}{RegexpStemmer}\PY{p}{(}\PY{l+s+s1}{\PYZsq{}}\PY{l+s+s1}{ing}\PY{l+s+s1}{\PYZsq{}}\PY{p}{)}
        \PY{n}{stemmer\PYZus{}regexp}\PY{o}{.}\PY{n}{stem}\PY{p}{(}\PY{l+s+s1}{\PYZsq{}}\PY{l+s+s1}{working}\PY{l+s+s1}{\PYZsq{}}\PY{p}{)}
\end{Verbatim}


\begin{Verbatim}[commandchars=\\\{\}]
{\color{outcolor}Out[{\color{outcolor}9}]:} 'work'
\end{Verbatim}
            
    \begin{Verbatim}[commandchars=\\\{\}]
{\color{incolor}In [{\color{incolor}10}]:}  \PY{n}{stemmer\PYZus{}regexp}\PY{o}{.}\PY{n}{stem}\PY{p}{(}\PY{l+s+s1}{\PYZsq{}}\PY{l+s+s1}{happiness}\PY{l+s+s1}{\PYZsq{}}\PY{p}{)}
\end{Verbatim}


\begin{Verbatim}[commandchars=\\\{\}]
{\color{outcolor}Out[{\color{outcolor}10}]:} 'happiness'
\end{Verbatim}
            
    \begin{Verbatim}[commandchars=\\\{\}]
{\color{incolor}In [{\color{incolor}11}]:}  \PY{n}{stemmer\PYZus{}regexp}\PY{o}{.}\PY{n}{stem}\PY{p}{(}\PY{l+s+s1}{\PYZsq{}}\PY{l+s+s1}{pairing}\PY{l+s+s1}{\PYZsq{}}\PY{p}{)}
\end{Verbatim}


\begin{Verbatim}[commandchars=\\\{\}]
{\color{outcolor}Out[{\color{outcolor}11}]:} 'pair'
\end{Verbatim}
            
    \hypertarget{snowballstemmer}{%
\section{SnowballStemmer}\label{snowballstemmer}}

SnowballStemmer is used to perform stemming in 13 languages other than
English. In order to perform stemming using SnowballStemmer, firstly, an
instance is created in the language in which stemming needs to be
performed. Then, using the stem() method, stemming is performed.

    \begin{Verbatim}[commandchars=\\\{\}]
{\color{incolor}In [{\color{incolor}12}]:} \PY{k+kn}{from} \PY{n+nn}{nltk}\PY{n+nn}{.}\PY{n+nn}{stem} \PY{k}{import} \PY{n}{SnowballStemmer}
         \PY{n}{SnowballStemmer}\PY{o}{.}\PY{n}{languages}
\end{Verbatim}


\begin{Verbatim}[commandchars=\\\{\}]
{\color{outcolor}Out[{\color{outcolor}12}]:} ('arabic',
          'danish',
          'dutch',
          'english',
          'finnish',
          'french',
          'german',
          'hungarian',
          'italian',
          'norwegian',
          'porter',
          'portuguese',
          'romanian',
          'russian',
          'spanish',
          'swedish')
\end{Verbatim}
            
    \begin{Verbatim}[commandchars=\\\{\}]
{\color{incolor}In [{\color{incolor}14}]:} \PY{n}{spanishstemmer}\PY{o}{=}\PY{n}{SnowballStemmer}\PY{p}{(}\PY{l+s+s1}{\PYZsq{}}\PY{l+s+s1}{spanish}\PY{l+s+s1}{\PYZsq{}}\PY{p}{)}
         \PY{n}{spanishstemmer}\PY{o}{.}\PY{n}{stem}\PY{p}{(}\PY{l+s+s1}{\PYZsq{}}\PY{l+s+s1}{comiendo}\PY{l+s+s1}{\PYZsq{}}\PY{p}{)}
\end{Verbatim}


\begin{Verbatim}[commandchars=\\\{\}]
{\color{outcolor}Out[{\color{outcolor}14}]:} 'com'
\end{Verbatim}
            
    \begin{Verbatim}[commandchars=\\\{\}]
{\color{incolor}In [{\color{incolor}15}]:} \PY{n}{frenchstemmer}\PY{o}{=}\PY{n}{SnowballStemmer}\PY{p}{(}\PY{l+s+s1}{\PYZsq{}}\PY{l+s+s1}{french}\PY{l+s+s1}{\PYZsq{}}\PY{p}{)}
         \PY{n}{frenchstemmer}\PY{o}{.}\PY{n}{stem}\PY{p}{(}\PY{l+s+s1}{\PYZsq{}}\PY{l+s+s1}{manger}\PY{l+s+s1}{\PYZsq{}}\PY{p}{)}
\end{Verbatim}


\begin{Verbatim}[commandchars=\\\{\}]
{\color{outcolor}Out[{\color{outcolor}15}]:} 'mang'
\end{Verbatim}
            
    \hypertarget{understanding-lemmatization}{%
\section{Understanding
lemmatization}\label{understanding-lemmatization}}

    Lemmatization is the process in which we transform the word into a form
with a different word category. The word formed after lemmatization is
entirely different. The built-in morphy() function is used for
lemmatization in WordNetLemmatizer. The inputted word is left unchanged
if it is not found in WordNet. In the argument, pos refers to the part
of speech category of the inputted word.

Consider an example of lemmatization in NLTK:

    \begin{Verbatim}[commandchars=\\\{\}]
{\color{incolor}In [{\color{incolor}18}]:} \PY{k+kn}{from} \PY{n+nn}{nltk}\PY{n+nn}{.}\PY{n+nn}{stem} \PY{k}{import} \PY{n}{WordNetLemmatizer}
         \PY{n}{lemmatizer\PYZus{}output}\PY{o}{=}\PY{n}{WordNetLemmatizer}\PY{p}{(}\PY{p}{)}
         \PY{n}{lemmatizer\PYZus{}output}\PY{o}{.}\PY{n}{lemmatize}\PY{p}{(}\PY{l+s+s1}{\PYZsq{}}\PY{l+s+s1}{working}\PY{l+s+s1}{\PYZsq{}}\PY{p}{)}
\end{Verbatim}


\begin{Verbatim}[commandchars=\\\{\}]
{\color{outcolor}Out[{\color{outcolor}18}]:} 'working'
\end{Verbatim}
            
    \begin{Verbatim}[commandchars=\\\{\}]
{\color{incolor}In [{\color{incolor}19}]:} \PY{n}{lemmatizer\PYZus{}output}\PY{o}{.}\PY{n}{lemmatize}\PY{p}{(}\PY{l+s+s1}{\PYZsq{}}\PY{l+s+s1}{working}\PY{l+s+s1}{\PYZsq{}}\PY{p}{,}\PY{n}{pos}\PY{o}{=}\PY{l+s+s1}{\PYZsq{}}\PY{l+s+s1}{v}\PY{l+s+s1}{\PYZsq{}}\PY{p}{)}
\end{Verbatim}


\begin{Verbatim}[commandchars=\\\{\}]
{\color{outcolor}Out[{\color{outcolor}19}]:} 'work'
\end{Verbatim}
            
    \begin{Verbatim}[commandchars=\\\{\}]
{\color{incolor}In [{\color{incolor}20}]:} \PY{n}{lemmatizer\PYZus{}output}\PY{o}{.}\PY{n}{lemmatize}\PY{p}{(}\PY{l+s+s1}{\PYZsq{}}\PY{l+s+s1}{works}\PY{l+s+s1}{\PYZsq{}}\PY{p}{)}
\end{Verbatim}


\begin{Verbatim}[commandchars=\\\{\}]
{\color{outcolor}Out[{\color{outcolor}20}]:} 'work'
\end{Verbatim}
            
    \hypertarget{developing-a-stemmer-for-non-english-language}{%
\section{Developing a stemmer for non-English
language}\label{developing-a-stemmer-for-non-english-language}}

    Polyglot is a software that is used to provide models called morfessor
models that are used to obtain morphemes from tokens. The Morpho
project's goal is to create unsupervised data-driven processes. The main
aim of the Morpho project is to focus on the creation of morphemes,
which is the smallest unit of syntax. Morphemes play an important role
in natural language processing. Morphemes are useful in automatic
recognition and the creation of language. With the help of the
vocabulary dictionaries of Polyglot, morfessor models on the 50,000
tokens of different languages were used.

Let's see the code for obtaining the language table using polyglot:

    Refer to Polyglot\_Package\_Usage

    \hypertarget{morphological-analyzer}{%
\section{Morphological analyzer}\label{morphological-analyzer}}

    Morphological analysis may be defined as the process of obtaining
grammatical information from tokens, given their suffix information.
Morphological analysis can be performed in three ways: morpheme-based
morphology (or anitem and arrangement approach), lexeme-based morphology
(or an item and process approach), and wordbased morphology (or a word
and paradigm approach). A morphological analyzer may be defined as a
program that is responsible for the analysis of the morphology of a
given input token. It analyzes a given token and generates morphological
information, such as gender, number, class, and so on, as an output.
We can determine the category of the word with the help of the following points:

• Morphological hints: The suffix's information helps us detect the category
of a word. For example, the -ness and –ment suffixes exist with nouns.

• Syntactic hints: Contextual information is conducive to determine the
category of a word. For example, if we have found the word that has the
noun category, then syntactic hints will be useful for determining whether
an adjective would appear before the noun or after the noun category.

• Semantic hints: A semantic hint is also useful for determining the word's
category. For example, if we already know that a word represents the name
of a location, then it will fall under the noun category.

• Open class: This is class of words that are not fixed, and their number keeps
on increasing every day, whenever a new word is added to their list. Words
in the open class are usually nouns. Prepositions are mostly in a closed class.
For example, there can be an unlimited number of words in the of Persons
list. So, it is an open class.

• Morphology captured by the Part of Speech tagset: The Part of Speech
tagset captures information that helps us perform morphology. For example,
the word plays would appear with the third person and a singular noun.

• Omorfi:Omorfi (Open morphology of Finnish) is a package that has
been licensed by GNU GPL version 3. It is used for performing numerous
tasks, such as language modeling, morphological analysis, rule-based
machine translation, information retrieval, statistical machine translation,
morphological segmentation, ontologies, and spell checking and correction


    \hypertarget{morphological-generator}{%
\section{Morphological generator}\label{morphological-generator}}

    A morphological generator is a program that performs the task of
morphological generation. Morphological generation may be considered an
opposite task of morphological analysis. Here, given the description of
a word in terms of number, category, stem, and so on, the original word
is retrieved. For example, if root = go, part of speech = verb, tense=
present, and if it occurs along with a third person and singular
subject, then a morphological generator would generate its surface form,
goes.

    There is a lot of Python-based software that performs morphological
analysis and generation. Some of them are as follows:
• ParaMorfo: It is used to perform morphological generation and analysis of
Spanish and Guarani nouns, adjectives, and verbs.

• HornMorpho: It is used for the morphological generation and analysis of
Oromo and Amharic nouns and verbs, as well as Tigrinya verbs.

• AntiMorfo: It is used for the morphological generation and analysis of
Quechua adjectives, verbs, and nouns, as well as Spanish verbs.

• MorfoMelayu: It is used for the morphological analysis of Malay words.
    
    Other examples of software that is used to perform morphological
analysis and generation are as follows:
• Morph is a morphological generator and analyzer for English for the RASP
system
• Morphy is a morphological generator, analyzer, and POS tagger for German
• Morphisto is a morphological generator and analyzer for German
• Morfette performs supervised learning (inflectional morphology) for Spanish
and French
    \hypertarget{search-engine}{%
\section{Search engine}\label{search-engine}}

    \begin{Verbatim}[commandchars=\\\{\}]
{\color{incolor}In [{\color{incolor}73}]:} \PY{k}{def} \PY{n+nf}{eliminatestopwords}\PY{p}{(}\PY{n+nb+bp}{self}\PY{p}{,}\PY{n+nb}{list}\PY{p}{)}\PY{p}{:}
             \PY{k}{return}\PY{p}{[} \PY{n}{word} \PY{k}{for} \PY{n}{word} \PY{o+ow}{in} \PY{n+nb}{list} \PY{k}{if} \PY{n}{word} \PY{o+ow}{not} \PY{o+ow}{in} \PY{n+nb+bp}{self}\PY{o}{.}\PY{n}{stopwords} \PY{p}{]}
\end{Verbatim}


    \begin{Verbatim}[commandchars=\\\{\}]
{\color{incolor}In [{\color{incolor}74}]:} \PY{k}{def} \PY{n+nf}{tokenize}\PY{p}{(}\PY{n+nb+bp}{self}\PY{p}{,}\PY{n}{string}\PY{p}{)}\PY{p}{:}
             \PY{n}{Str}\PY{o}{=}\PY{n+nb+bp}{self}\PY{o}{.}\PY{n}{clean}\PY{p}{(}\PY{n+nb}{str}\PY{p}{)}
             \PY{n}{Words}\PY{o}{=}\PY{n+nb}{str}\PY{o}{.}\PY{n}{split}\PY{p}{(}\PY{l+s+s2}{\PYZdq{}}\PY{l+s+s2}{ }\PY{l+s+s2}{\PYZdq{}}\PY{p}{)}
             \PY{k}{return} \PY{p}{[}\PY{n+nb+bp}{self}\PY{o}{.}\PY{n}{stemmer}\PY{o}{.}\PY{n}{stem}\PY{p}{(}\PY{n}{word}\PY{p}{,}\PY{l+m+mi}{0}\PY{p}{,}\PY{n+nb}{len}\PY{p}{(}\PY{n}{word}\PY{p}{)}\PY{o}{\PYZhy{}}\PY{l+m+mi}{1}\PY{p}{)} \PY{k}{for} \PY{n}{word} \PY{o+ow}{in} \PY{n}{words}\PY{p}{]}
\end{Verbatim}


    \begin{Verbatim}[commandchars=\\\{\}]
{\color{incolor}In [{\color{incolor}75}]:} \PY{k}{def} \PY{n+nf}{obtainvectorkeywordindex}\PY{p}{(}\PY{n+nb+bp}{self}\PY{p}{,} \PY{n}{documentList}\PY{p}{)}\PY{p}{:}
             \PY{n}{vocabstring} \PY{o}{=} \PY{l+s+s2}{\PYZdq{}}\PY{l+s+s2}{\PYZdq{}}\PY{o}{.}\PY{n}{join}\PY{p}{(}\PY{n}{documentList}\PY{p}{)}
             \PY{n}{vocablist} \PY{o}{=} \PY{n+nb+bp}{self}\PY{o}{.}\PY{n}{parser}\PY{o}{.}\PY{n}{tokenise}\PY{p}{(}\PY{n}{vocabstring}\PY{p}{)}
             \PY{n}{vocablist} \PY{o}{=} \PY{n+nb+bp}{self}\PY{o}{.}\PY{n}{parser}\PY{o}{.}\PY{n}{eliminatestopwords}\PY{p}{(}\PY{n}{vocablist}\PY{p}{)}
             \PY{n}{uniqueVocablist} \PY{o}{=} \PY{n}{util}\PY{o}{.}\PY{n}{removeDuplicates}\PY{p}{(}\PY{n}{vocablist}\PY{p}{)}
             \PY{n}{vectorIndex}\PY{o}{=}\PY{p}{\PYZob{}}\PY{p}{\PYZcb{}}
             \PY{n}{offset}\PY{o}{=}\PY{l+m+mi}{0}
             \PY{k}{for} \PY{n}{word} \PY{o+ow}{in} \PY{n}{uniqueVocablist}\PY{p}{:}
                 \PY{n}{vectorIndex}\PY{p}{[}\PY{n}{word}\PY{p}{]}\PY{o}{=}\PY{n}{offset}
                 \PY{n}{offset}\PY{o}{+}\PY{o}{=}\PY{l+m+mi}{1}
             \PY{k}{return} \PY{n}{vectorIndex}
\end{Verbatim}


    \begin{Verbatim}[commandchars=\\\{\}]
{\color{incolor}In [{\color{incolor}76}]:} \PY{k}{def} \PY{n+nf}{constructVector}\PY{p}{(}\PY{n+nb+bp}{self}\PY{p}{,} \PY{n}{wordString}\PY{p}{)}\PY{p}{:}
             \PY{n}{Vector\PYZus{}val} \PY{o}{=} \PY{p}{[}\PY{l+m+mi}{0}\PY{p}{]} \PY{o}{*} \PY{n+nb}{len}\PY{p}{(}\PY{n+nb+bp}{self}\PY{o}{.}\PY{n}{vectorKeywordIndex}\PY{p}{)}
             \PY{n}{tokList} \PY{o}{=} \PY{n+nb+bp}{self}\PY{o}{.}\PY{n}{parser}\PY{o}{.}\PY{n}{tokenize}\PY{p}{(}\PY{n}{tokString}\PY{p}{)}
             \PY{n}{tokList} \PY{o}{=} \PY{n+nb+bp}{self}\PY{o}{.}\PY{n}{parser}\PY{o}{.}\PY{n}{eliminatestopwords}\PY{p}{(}\PY{n}{tokList}\PY{p}{)}
             \PY{k}{for} \PY{n}{word} \PY{o+ow}{in} \PY{n}{toklist}\PY{p}{:}
                 \PY{n}{vector}\PY{p}{[}\PY{n+nb+bp}{self}\PY{o}{.}\PY{n}{vectorKeywordIndex}\PY{p}{[}\PY{n}{word}\PY{p}{]}\PY{p}{]} \PY{o}{+}\PY{o}{=} \PY{l+m+mi}{1}\PY{p}{;}
             \PY{k}{return} \PY{n}{vector}
\end{Verbatim}


    \begin{Verbatim}[commandchars=\\\{\}]
{\color{incolor}In [{\color{incolor}77}]:} \PY{k}{def} \PY{n+nf}{cosine}\PY{p}{(}\PY{n}{vec1}\PY{p}{,} \PY{n}{vec2}\PY{p}{)}\PY{p}{:}
             \PY{k}{return} \PY{n+nb}{float}\PY{p}{(}\PY{n}{dot}\PY{p}{(}\PY{n}{vec1}\PY{p}{,}\PY{n}{vec2}\PY{p}{)} \PY{o}{/} \PY{p}{(}\PY{n}{norm}\PY{p}{(}\PY{n}{vec1}\PY{p}{)} \PY{o}{*} \PY{n}{norm}\PY{p}{(}\PY{n}{vec2}\PY{p}{)}\PY{p}{)}\PY{p}{)}
\end{Verbatim}


    \begin{Verbatim}[commandchars=\\\{\}]
{\color{incolor}In [{\color{incolor}78}]:} \PY{k}{def} \PY{n+nf}{searching}\PY{p}{(}\PY{n+nb+bp}{self}\PY{p}{,}\PY{n}{searchinglist}\PY{p}{)}\PY{p}{:}
             \PY{n}{askVector} \PY{o}{=} \PY{n+nb+bp}{self}\PY{o}{.}\PY{n}{buildQueryVector}\PY{p}{(}\PY{n}{searchinglist}\PY{p}{)}
             \PY{n}{ratings} \PY{o}{=} \PY{p}{[}\PY{n}{util}\PY{o}{.}\PY{n}{cosine}\PY{p}{(}\PY{n}{askVector}\PY{p}{,} \PY{n}{textVector}\PY{p}{)} \PY{k}{for} \PY{n}{textVector} \PY{o+ow}{in} \PY{n+nb+bp}{self}\PY{o}{.}\PY{n}{documentVectors}\PY{p}{]}
             \PY{n}{ratings}\PY{o}{.}\PY{n}{sort}\PY{p}{(}\PY{n}{reverse}\PY{o}{=}\PY{k+kc}{True}\PY{p}{)}
             \PY{k}{return} \PY{n}{ratings}
\end{Verbatim}


    \begin{Verbatim}[commandchars=\\\{\}]
{\color{incolor}In [{\color{incolor}79}]:} \PY{k+kn}{import} \PY{n+nn}{sys}
\end{Verbatim}


    \begin{Verbatim}[commandchars=\\\{\}]
{\color{incolor}In [{\color{incolor}80}]:} \PY{k}{try}\PY{p}{:}
             \PY{k+kn}{from} \PY{n+nn}{nltk} \PY{k}{import} \PY{n}{wordpunct\PYZus{}tokenize}
             \PY{k+kn}{from} \PY{n+nn}{nltk}\PY{n+nn}{.}\PY{n+nn}{corpus} \PY{k}{import} \PY{n}{stopwords}
         \PY{k}{except} \PY{n+ne}{ImportError}\PY{p}{:}
             \PY{n+nb}{print}\PY{p}{(} \PY{l+s+s1}{\PYZsq{}}\PY{l+s+s1}{Error has occured}\PY{l+s+s1}{\PYZsq{}}\PY{p}{)}
\end{Verbatim}


    \begin{Verbatim}[commandchars=\\\{\}]
{\color{incolor}In [{\color{incolor}81}]:} \PY{k}{def} \PY{n+nf}{\PYZus{}calculate\PYZus{}languages\PYZus{}ratios}\PY{p}{(}\PY{n}{text}\PY{p}{)}\PY{p}{:}
             \PY{n}{languages\PYZus{}ratios} \PY{o}{=} \PY{p}{\PYZob{}}\PY{p}{\PYZcb{}}
             \PY{n}{tok} \PY{o}{=} \PY{n}{wordpunct\PYZus{}tokenize}\PY{p}{(}\PY{n}{text}\PY{p}{)}
             \PY{n}{wor} \PY{o}{=} \PY{p}{[}\PY{n}{word}\PY{o}{.}\PY{n}{lower}\PY{p}{(}\PY{p}{)} \PY{k}{for} \PY{n}{word} \PY{o+ow}{in} \PY{n}{tok}\PY{p}{]}
             \PY{k}{for} \PY{n}{language} \PY{o+ow}{in} \PY{n}{stopwords}\PY{o}{.}\PY{n}{fileids}\PY{p}{(}\PY{p}{)}\PY{p}{:}
                 \PY{n}{stopwords\PYZus{}set} \PY{o}{=} \PY{n+nb}{set}\PY{p}{(}\PY{n}{stopwords}\PY{o}{.}\PY{n}{words}\PY{p}{(}\PY{n}{language}\PY{p}{)}\PY{p}{)}
                 \PY{n}{words\PYZus{}set} \PY{o}{=} \PY{n+nb}{set}\PY{p}{(}\PY{n}{wor}\PY{p}{)}
                 \PY{n}{common\PYZus{}elements} \PY{o}{=} \PY{n}{words\PYZus{}set}\PY{o}{.}\PY{n}{intersection}\PY{p}{(}\PY{n}{stopwords\PYZus{}set}\PY{p}{)}
                 \PY{n}{languages\PYZus{}ratios}\PY{p}{[}\PY{n}{language}\PY{p}{]} \PY{o}{=} \PY{n+nb}{len}\PY{p}{(}\PY{n}{common\PYZus{}elements}\PY{p}{)}
                 \PY{k}{return} \PY{n}{languages\PYZus{}ratios}
\end{Verbatim}


    \begin{Verbatim}[commandchars=\\\{\}]
{\color{incolor}In [{\color{incolor}82}]:} \PY{k}{def} \PY{n+nf}{detect\PYZus{}language}\PY{p}{(}\PY{n}{text}\PY{p}{)}\PY{p}{:}
             \PY{n}{ratios} \PY{o}{=} \PY{n}{\PYZus{}calculate\PYZus{}languages\PYZus{}ratios}\PY{p}{(}\PY{n}{text}\PY{p}{)}
             \PY{n}{most\PYZus{}rated\PYZus{}language} \PY{o}{=} \PY{n+nb}{max}\PY{p}{(}\PY{n}{ratios}\PY{p}{,} \PY{n}{key}\PY{o}{=}\PY{n}{ratios}\PY{o}{.}\PY{n}{get}\PY{p}{)}
             \PY{k}{return} \PY{n}{most\PYZus{}rated\PYZus{}language}
\end{Verbatim}


    \begin{Verbatim}[commandchars=\\\{\}]
{\color{incolor}In [{\color{incolor}93}]:} \PY{n}{text} \PY{o}{=} \PY{l+s+s2}{\PYZdq{}\PYZdq{}\PYZdq{}}
         \PY{l+s+s2}{الثالث على التوالي عن التقدم الى داخل مدينة تكريت بسبب}
         \PY{l+s+s2}{انتشار قناصي التنظيم الذي ي}\PY{l+s+s2}{\PYZdq{}\PYZdq{}\PYZdq{}}
\end{Verbatim}


    \begin{Verbatim}[commandchars=\\\{\}]
{\color{incolor}In [{\color{incolor}94}]:} \PY{n}{language} \PY{o}{=} \PY{n}{detect\PYZus{}language}\PY{p}{(}\PY{n}{text}\PY{p}{)}
\end{Verbatim}


    \begin{Verbatim}[commandchars=\\\{\}]
{\color{incolor}In [{\color{incolor}95}]:} \PY{n+nb}{print}\PY{p}{(}\PY{n}{language}\PY{p}{)}
\end{Verbatim}


    \begin{Verbatim}[commandchars=\\\{\}]
arabic

    \end{Verbatim}


    % Add a bibliography block to the postdoc
    
    
    
    \end{document}
